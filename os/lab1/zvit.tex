\documentclass[12pt,a4paper]{article}

\usepackage{polyglossia}
\usepackage[margin=1in]{geometry}

\setdefaultlanguage{ukrainian}
\setotherlanguage{english}
\setmainfont[Ligatures={TeX}]{Liberation Serif}
\setsansfont{Liberation Sans}
\setmonofont{Liberation Mono}

\begin{document}
\begin{titlepage}
	{\centering
	\Large НАЦІОНАЛЬНИЙ ТЕХНІЧНИЙ УНІВЕРСИТЕТ УКРАЇНИ\par
	\large «Київський політехнічний інститут імені Ігоря Сікорського»\par
	\vspace{1cm}
	Кафедра програмного забезпечення комп'ютерних систем\par
	\vspace{1cm}
	\normalsize Лабораторна робота \textnumero1\par
	із дисципліни «Операційні системи»\par
	\vspace{2cm}
	на тему: \textbf{ВСТАНОВЛЕННЯ І ВИКОРИСТАННЯ ORACLE VM VIRTUALBOX}}
	\vspace{1cm}
	\begin{flushright}
		Виконав:

		студент 2 курсу ФПМ групи КП-52

		\textit{Комар Григорій Миколайович}

		\vspace{1cm}

		Прийняла:

		\textit{к.т.н., ст. викл. Рибачок Наталія Антонівна}

		\textit{“19” вересня 2016р.}

		\vspace{1cm}
		\begin{tabular}{|l|l|}
			\hline
			&Бали\\ \hline
			Якість виконання&\\ \hline
			Відповіді на питанняя&\\ \hline
			Оформлення звіту&\\ \hline
			Термін здачі&\\ \hline
			\multicolumn{1}{|r|}{Сумарний бал}&\\ \hline
		\end{tabular}
	\end{flushright}
	\vfill
	\centering КИЇВ — 2016
\end{titlepage}
\section{Завдання на лабораторну роботу}
\begin{enumerate}
	\item Встановіть Oracle VM VirtualBox (https://www.virtualbox.org/wiki/Downloads/)
	\item Створіть ВМ. ВМ повинна мати назву «Ваше прізвище\_ОС»
	\item Встановіть на ВМ операційну систему за вибором
	\item При налаштуванні ВМ потрібно підключити і налаштувати:
		\begin{itemize}
			\item двонаправлений буфер обміну між ВМ та основною ОС
			\item CD/DVD (образ, що використовувався для встановлення ОС)
			\item мережу через NAT
			\item USB-накопичувачі
			\item спільні теки
			\item встановіть порядок завантаження ВМ: жорсткий диск, CD/DVD
		\end{itemize}
	\item Дослідіть в яких папках знаходяться файли ВМ та для чого вони використовуються:
		\begin{itemize}
			\item Файли віртуального ЖД, файли конфігурації
			\item Зробіть знімок стану ВМ. Які зміни відбуваються у файловій системі?
			\item Збережіть стан ВМ (Файл-Закрити-Зберегти стан машини). Які файли при цьому
			створюються? Відновіть роботу вашої ВМ
			\item Створіть файл експорту ВМ
		\end{itemize}
	\item Вкажіть, які можливості надає двонаправлений буфер обміну між встановленою ВМ та
		основною ОС та чи працює функція drag'n'drop
\end{enumerate}
\end{document}
